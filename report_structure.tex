\documentclass[12pt,a4paper]{article}
\usepackage[utf8]{inputenc}
\usepackage[spanish]{babel}
\usepackage{graphicx}
\usepackage{hyperref}

\title{Informe Técnico: Desarrollo e Implementación de un Pipeline CI/CD Seguro con IA}
\author{Nombre del Estudiante}
\date{Diciembre 2025}

\begin{document}

\maketitle

\begin{abstract}
Este informe detalla la implementación de un sistema de detección automática de vulnerabilidades integrado en un pipeline CI/CD, utilizando técnicas de Minería de Datos (SEMMA) y algoritmos de clasificación supervisada (Random Forest/SVM), cumpliendo con los principios de DevSecOps.
\end{abstract}

\section{Introducción}
La seguridad en el desarrollo de software es crítica... (Contexto del proyecto).

\section{Metodología SEMMA}
\subsection{Sample (Muestreo)}
Descripción de los datasets utilizados (ZeoVan, Sintéticos)...

\subsection{Explore (Exploración)}
Análisis exploratorio de datos, distribución de clases...

\subsection{Modify (Modificación)}
Ingeniería de características:
\begin{itemize}
    \item TF-IDF
    \item Complejidad Ciclomática
    \item Profundidad AST
    \item Llamadas a funciones peligrosas
\end{itemize}

\subsection{Model (Modelado)}
Descripción de los algoritmos seleccionados (Random Forest, SVM) y la optimización de hiperparámetros (GridSearch)...

\subsection{Assess (Evaluación)}
Resultados obtenidos:
\begin{itemize}
    \item Precisión: XX\%
    \item Recall: XX\%
    \item F1-Score: XX\%
\end{itemize}
(Incluir capturas de curvas de aprendizaje).

\section{Arquitectura del Pipeline CI/CD}
Descripción de las 3 etapas del flujo en GitHub Actions...
\begin{enumerate}
    \item Revisión de Seguridad
    \item Pruebas Unitarias
    \item Despliegue
\end{enumerate}

\section{Resultados y Discusión}
Análisis de la efectividad del modelo en casos reales...

\section{Conclusiones}
El sistema demuestra que es posible integrar IA clásica en pipelines DevOps...

\end{document}
