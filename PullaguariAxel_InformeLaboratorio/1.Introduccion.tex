\chapter{Introducción}
El crecimiento exponencial del código fuente en los ecosistemas de software modernos ha hecho que la revisión manual de seguridad sea inviable. Las vulnerabilidades críticas, como las listadas en el OWASP Top 10, persisten debido a la falta de herramientas automatizadas que puedan aprender y adaptarse a nuevos patrones de ataque. En este contexto, la Minería de Datos (Data Mining) y el Aprendizaje Automático (Machine Learning) emergen como soluciones poderosas para detectar patrones de código inseguro de manera proactiva.

Este proyecto implementa un sistema de detección de vulnerabilidades utilizando la metodología estándar de la industria para minería de datos: **SEMMA** (Sample, Explore, Modify, Model, Assess). A diferencia de las herramientas estáticas tradicionales (SAST) que se basan en reglas rígidas, nuestro enfoque entrena modelos predictivos (Random Forest y SVM) con miles de ejemplos de código real y sintético. Esto permite no solo identificar vulnerabilidades conocidas, sino también inferir riesgos basados en la estructura y complejidad del código, proporcionando una capa de seguridad inteligente e integrada en el ciclo de vida de desarrollo (DevSecOps).
