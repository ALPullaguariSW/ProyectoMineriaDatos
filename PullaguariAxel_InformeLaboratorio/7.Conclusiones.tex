\chapter{Conclusiones}

La implementación de la metodología SEMMA permitió estructurar un proceso robusto de minería de datos aplicado a la ciberseguridad. Se logró construir un escáner de vulnerabilidades capaz de procesar múltiples lenguajes de programación y detectar riesgos críticos con una precisión cercana al 100\%.

El uso de algoritmos de Random Forest, combinado con una ingeniería de características profunda (AST, Complejidad), demostró ser altamente efectivo. El proyecto no solo cumple con los objetivos académicos de aplicar minería de datos, sino que resulta en una herramienta práctica (MVP) que puede integrarse en pipelines de CI/CD reales para mejorar la postura de seguridad de cualquier desarrollo de software.
