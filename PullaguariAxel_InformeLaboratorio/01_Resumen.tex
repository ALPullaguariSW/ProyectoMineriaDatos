\begin{abstract}
    Este informe detalla la implementación de un sistema de detección de vulnerabilidades de software basado en la metodología de Minería de Datos SEMMA (Sample, Explore, Modify, Model, Assess). El proyecto integra un pipeline automatizado que recolecta código fuente de repositorios masivos (GitHub), preprocesa los datos extrayendo características léxicas y sintácticas (TF-IDF, Complejidad Ciclomática, AST), y entrena modelos de aprendizaje automático (Random Forest y SVM) para clasificar código como seguro o vulnerable. Se utilizó un dataset híbrido compuesto por datos sintéticos y más de 180,000 muestras reales minadas de proyectos Open Source. Los resultados demuestran una precisión del 99.9\% en la detección de vulnerabilidades críticas (OWASP Top 10), validando la eficacia de la minería de datos en la mejora de la seguridad del software.
\end{abstract}