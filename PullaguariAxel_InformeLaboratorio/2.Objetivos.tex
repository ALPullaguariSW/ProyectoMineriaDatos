\chapter{Objetivos}

\section{Objetivo General}
Desarrollar e implementar un sistema de detección automática de vulnerabilidades de software utilizando técnicas de Minería de Datos y la metodología SEMMA, capaz de clasificar código fuente como seguro o vulnerable con alta precisión.

\section{Objetivos Específicos}
\begin{itemize}
    \item \textbf{Sample (Muestreo)}: Recopilar un dataset masivo y representativo de código fuente mediante la minería de repositorios Open Source (GitHub) y la generación de datos sintéticos.
    \item \textbf{Explore (Exploración)}: Analizar la distribución de vulnerabilidades y características del código mediante técnicas de Análisis Exploratorio de Datos (EDA).
    \item \textbf{Modify (Modificación)}: Preprocesar el código fuente extrayendo características relevantes como vectores TF-IDF, complejidad ciclomática y profundidad del árbol sintáctico (AST).
    \item \textbf{Model (Modelado)}: Entrenar y optimizar modelos de aprendizaje automático (Random Forest, SVM) para la clasificación binaria de vulnerabilidades.
    \item \textbf{Assess (Evaluación)}: Validar el rendimiento de los modelos mediante métricas de precisión, recall, curvas de aprendizaje y pruebas con código real.
\end{itemize}